\documentclass[]{scrartcl}
\usepackage{inputenc}
\usepackage[german]{babel}
\usepackage{hyperref}
\usepackage{amsmath}
\usepackage{amsfonts}

%opening
\title{Mathematische Knobeleien}
\subtitle{Problemsammlung}
\author{Mathematik - Verständlich gemacht!\footnote{Email: \href{mailto:kontakt@mschulte-mathematik.ruhr}{kontakt@mschulte-mathematik.ruhr}}}

\begin{document}

\maketitle

\noindent 
In diesem Dokument werden die Problemstellungen aus den 
"Mathematischen Knobeleien"\ gesammelt. Lösungen,
Ergänzungen und Quellenangaben finden sich in den einzelnen
Dateien zu den Knobeleien.

\section*{Problem 1 - Gezinkte Würfel}
Zwei Würfel werden gezinkt. Beim ersten Würfel steigt die Wahrscheinlichkeit, eine 1 zu
werfen, auf $\frac{1}{5}$; beim zweiten Würfel steigt hingegen die Wahrscheinlichkeit, eine 6
zu werfen, auf $\frac{1}{5}$.

\noindent 
Um wie viel ändert sich die Wahrscheinlichkeit im Vergleich mit zwei fairen Würfeln, mit den
gezinkten Würfeln die Augensumme 7 zu werfen?

\section*{Problem 2 - Logische Logarithmen}
Es seien $a,b,c \in (0,\infty) \setminus \{1\}$. Bestimme den Wert von
$$
x = \frac{1}{1+\log_a(bc)} + \frac{1}{1+\log_b(ac)} + \frac{1}{1+\log_c(ab)}.
$$

\section*{Problem 3 - Ein Dreieck im Trapez}
Es sei $T:=ABCD$ ein gleichschenkliges Trapez mit den 
Seiten\footnote{Wir identifizieren Seitennamen mit ihren 
Längen.} $a:=AB$, $b:=BC$, $c:=CD$ und $d:=DA$. Seine Höhe
bezeichnen wir mit $h_T$. Nun ziehen wir die
beiden Diagonalen $AC$ und $DB$ ein und bezeichnen ihren 
Schnittpunkt mit $S$. Von $C$ und $D$ fällen wir jeweils das
Lot auf $a$; die entstehenden Schnittpunkte nennen wir $S_C$ bzw.
$S_D$. Es entsteht ein Dreieck $\Delta := S_DS_CS$. 
Bestimme den Flächeninhalt von $\Delta$.

\end{document}
