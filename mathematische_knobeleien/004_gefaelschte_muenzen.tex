\documentclass[ngerman]{scrartcl}
\usepackage{inputenc}
\usepackage{hyperref}
\usepackage{amsmath}
\usepackage{amsfonts}

%opening
\title{Mathematische Knobeleien}
\subtitle{Teil 4 -- Gefälschte Münzen}
\author{Mathematik - Verständlich gemacht!\footnote{Email: \href{mailto:kontakt@mschulte-mathematik.ruhr}{kontakt@mschulte-mathematik.ruhr}}}

\begin{document}

\maketitle

\section*{Problemstellung}
Zwei Münzen $M_1$ und $M_2$ werden geworfen. Dabei ist $M_1$ eine
faire Münze, während $M_2$ derart gefälscht wurde, dass die 
Wahrscheinlichkeit, mit ihr zweimal hintereinander \textit{Kopf} zu 
werfen, genauso groß ist, wie die Wahrscheinlichkeit, mit $M_1$ 
\textit{Zahl} zu werfen. Wie groß ist die Wahrscheinlichkeit, mit 
$M_2$ \textit{Kopf} zu werfen?

\section*{Lösung}
xxx

\section*{Quellen}
Die Problemstellung entstammt aus

\centering
\textit{Heinrich Hemme: Heureka! Mathematische Rätsel 2023: Tageskalender mit Lösungen.
	2022. München. Anaconda Verlag. ISBN 978-3-7306-1077-0.}

\raggedright
Es handelt sich um das Rätsel vom 23.03.2023.

\end{document}
