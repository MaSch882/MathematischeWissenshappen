\documentclass[]{scrartcl}
\usepackage{inputenc}
\usepackage[german]{babel}
\usepackage{hyperref}
\usepackage{amsmath}
\usepackage{amsfonts}
\usepackage{stmaryrd}

%opening
\title{Mathematische Knobeleien}
\subtitle{Teil 2 -- Logische Logarithmen}
\author{Mathematik - Verständlich gemacht!\footnote{Email: \href{mailto:kontakt@mschulte-mathematik.ruhr}{kontakt@mschulte-mathematik.ruhr}}}

\begin{document}

\maketitle

\section*{Problemstellung}
Es seien $a,b,c \in (0,\infty) \setminus \{1\}$. Bestimme den Wert von
$$
x = \frac{1}{1+\log_a(bc)} + \frac{1}{1+\log_b(ac)} + \frac{1}{1+\log_c(ab)}.
$$

\section*{Lösung}
Wir benutzen die Formel 
\begin{align*}
	\log_a(x) = \frac{\log x}{\log a}, a \in (0,\infty) \setminus \{1\},
	\tag{$\ast$}
	\label{eq_nat_log}
\end{align*}
wobei $\log x$ der natürliche Logarithmus von $x$ ist.
Damit ergibt sich 
\begin{align*}
	x 
	& \overset{\eqref{eq_nat_log}}{=}
	\frac{1}{1+\frac{\log(bc)}{\log a}} +
	\frac{1}{1+\frac{\log(ac)}{\log b}} +
	\frac{1}{1+\frac{\log(ab)}{\log c}}
	\\ &= 
	\frac{\log a}{\log a + \log(bc)} +
	\frac{\log b}{\log b + \log(ac)} +
	\frac{\log a}{\log c + \log(ab)}.
\end{align*}
Nun benutzen wir das Logarithmusgesetz
\begin{align*}
	\log x + \log y = \log(xy)
	\tag{$\ast\ast$}, x,y > 0,
	\label{eq_log_gesetz}
\end{align*}
und erhalten
\begin{align*}
	x 
	& \overset{\eqref{eq_log_gesetz}}{=}
	\frac{\log a}{\log(abc)} +
	\frac{\log b}{\log(abc)} +
	\frac{\log c}{\log(abc)} 
	\overset{\eqref{eq_log_gesetz}}{=}
	\frac{\log(abc)}{\log(abc)} = 1.
\end{align*}

\section*{Ergänzungen}
\begin{itemize}
	\item[1.]
	{
	Die Formel \eqref{eq_nat_log} ergibt sich wie folgt:
	\begin{align*}
		b = \log_a(x) 
		& \Leftrightarrow
		a^b = a^{\log_a(x)} = x
		\\ & \Leftrightarrow
		\log(a^b) =\log x
		\\ & \Leftrightarrow
		b \cdot \log a = \log x
		\\ & \Leftrightarrow
		b = \frac{\log x}{\log a}.
	\end{align*}
	Hierbei dürfen wir äquivalent umformen, da Potenzfunktionen 
	und die natürliche Logarithmusfunktion bijektiv sind.
	}
	\item[2.]
	{
	Die Formel \eqref{eq_log_gesetz} sieht man beispielsweise 
	wie folgt ein:
	\begin{align*}
		\log x + \log y = \log(xy) 
		& \Leftrightarrow
		e^{\log x + \log y} = e^{\log(xy)}
		\\ & \Leftrightarrow
		e^{\log x} \cdot e^{\log y} = xy 
		\\ & \Leftrightarrow
		x \cdot y = xy. ~ \checkmark
	\end{align*}
	Hier dürfen wir äquivalent umformen, da die natürliche 
	Exponentialfunktion bijektiv ist und die Funktionalgleichung
	$$
	f(x+y) = f(x) \cdot f(y)
	$$
	erfüllt.
	}
	\item[3.]
	{
	Die Voraussetzung $a,b,c \neq 1$ erklärt sich dadurch, dass
	Logarithmen zur Basis 1 keinen Sinn ergeben. So wäre 
	$\log_1(x)$ die Lösung der Gleichung $1^x = b$; hier gibt 
	es für $b \neq 1$ keine Lösung und für $b=1$ unendlich viele
	Lösungen.
	\newline 
	Algebraisch können wir dies auch mit Formel \eqref{eq_nat_log}
	einsehen:
	$$
	b = \log_1(x) \Leftrightarrow
	b = \frac{\log x}{\log 1} = \frac{\log x}{0} = \infty. 
	~ \lightning
	$$
	}
\end{itemize}

\section*{Quellen}
Die Problemstellung entstammt aus

\centering
\textit{Heinrich Hemme: Heureka! Mathematische Rätsel 2023: Tageskalender mit Lösungen.
	2022. München. Anaconda Verlag. ISBN 978-3-7306-1077-0.}

\raggedright
Es handelt sich um das Rätsel vom 15.03.2023.

\end{document}
