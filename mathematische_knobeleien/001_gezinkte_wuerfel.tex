\documentclass[]{scrartcl}
\usepackage{inputenc}
\usepackage[german]{babel}
\usepackage{hyperref}
\usepackage{amsmath}
\usepackage{amsfonts}

%opening
\title{Mathematische Knobeleien}
\subtitle{Teil 1 -- Gezinkte Würfel}
\author{Mathematik - Verständlich gemacht!\footnote{Email: \href{mailto:kontakt@mschulte-mathematik.ruhr}{kontakt@mschulte-mathematik.ruhr}}}

\begin{document}

\maketitle

\section*{Problemstellung}
Zwei Würfel werden gezinkt. Beim ersten Würfel steigt die Wahrscheinlichkeit, eine 1 zu
werfen, auf $\frac{1}{5}$; beim zweiten Würfel steigt hingegen die Wahrscheinlichkeit, eine 6
zu werfen, auf $\frac{1}{5}$.

\noindent 
Um wie viel ändert sich die Wahrscheinlichkeit im Vergleich mit zwei fairen Würfeln, mit den
gezinkten Würfeln die Augensumme 7 zu werfen?

\section*{Lösung}
Zunächst berechnen wir die Wahrscheinlichkeit, mit zwei fairen Würfeln eine 7 zu werfen.
Da wir faire Würfel betrachten, ist die Augenzahl jedes Wurfs gleichverteilt auf $\{1,\ldots,6\}$
und die Würfe sind unabhängig voneinander. Es sei $X$ die Zufallsvariable, die die Augensumme
der beiden Würfe beschreibt. Ferner sei $\mathbb{P}$ das Produktmaß zur
Laplace-Verteilung auf $\{1,\ldots,6\}^2$. Dann gilt
$$
\mathbb{P}(X=7) = \mathbb{P}(\{ (1,6), (2,5), (3,4), (4,3), (5,2), (6,1) \}) = 6 \cdot
\frac{1}{6} \cdot \frac{1}{6} = \frac{1}{6}.
$$
Nun modellieren wir die beiden gezinkten Würfel.

\begin{itemize}
	\item[1.]
	{
	Es sei $\mathbb{P}_1$ das Wahrscheinlichkeitsmaß zur Verteilung des ersten gezinkten Würfels.
	Dann gilt bei Annahme einer Gleichverteilung von $\omega \in \{2,\ldots,6\}$:
	$$
	\mathbb{P}_1(\{1\}) = \frac{1}{5}, ~ \mathbb{P}_1(\{\omega\}) = \frac{1-1/5}{5} = \frac{4}{25},
	~ \omega \in \{2,\ldots,6\}.
	$$
	}
	\item[2.]
	{
	Analog ergibt sich für den zweiten Würfel mit (analog konstruiertem) Wahrscheinlichkeitsmaß
	$\mathbb{P}_2$:
	$$
	\mathbb{P}_2(\{6\}) = \frac{1}{5}, ~ \mathbb{P}_1(\{\omega\}) = \frac{1-1/5}{5} = \frac{4}{25},
	~ \omega \in \{1,\ldots,5\}.
	$$
	}
\end{itemize}
Nun betrachten wir das Produktexperiment bzw. das Produktmaß aus beiden Würfen. Hierzu sei $Y$ 
die Zufallsvariable, die die Augensumme beider gezinkten Würfel beschreibt und $\mathbb{P}_Y$ das
Produktmaß aus $\mathbb{P}_1$ und $\mathbb{P}_2$, d.h.
$$
\mathbb{P}_Y = \mathbb{P}_1 \otimes \mathbb{P}_2.
$$
Damit folgt
$$
\mathbb{P}_Y(Y=7) =
\mathbb{P}_Y(\{(1,6),(2,5),(3,4),(4,3),(5,2),(6,1)\}).
$$
Unter Ausnutzung der Rechenregeln für Wahrscheinlichkeitsmaße und disjunkte Mengen folgt
$$
\mathbb{P}_Y(Y=7) =
\mathbb{P}_Y(\{(1,6)\}) + \ldots + \mathbb{P}_Y(\{(6,1)\}).
$$
Hier benutzen wir nun die Eigenschaft des Produktmaßes 
$$
(\mathbb{P}_1 \otimes \mathbb{P}_2)(A) = \mathbb{P}_1(A) \cdot \mathbb{P}_2(A)
$$
und erhalten
$$
\mathbb{P}_Y(Y=7) =
\mathbb{P}_1(\{1\}) \cdot \mathbb{P}_2(\{6\}) + \ldots + \mathbb{P}_1(\{6\}) \cdot
\mathbb{P}_2(\{1\}) = \frac{1}{5} \cdot \frac{1}{5} + 5 \cdot \frac{4}{25} \cdot \frac{4}{25}
= \frac{126}{750}.
$$
Wegen $\frac{1}{6} = \frac{125}{750}$ ist die Antwort auf die eingangs formulierte Fragestellung:
\newline 
\textbf{Die Wahrscheinlichkeit, mit den gezinkten Würfeln eine 7 zu werfen, erhöht sich um} 
$\pmb{\frac{1}{750} \approx 0.13\%}$.

\section*{Quellen}
Die Problemstellung entstammt aus

\centering
	\textit{Heinrich Hemme: Heureka! Mathematische Rätsel 2023: Tageskalender mit Lösungen.
	2022. München. Anaconda Verlag. ISBN 978-3-7306-1077-0.}

\raggedright
Es handelt sich um das Rätsel vom xx.02.2023.
\end{document}
