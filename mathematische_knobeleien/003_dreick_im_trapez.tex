\documentclass[]{scrartcl}
\usepackage{inputenc}
\usepackage[german]{babel}
\usepackage{hyperref}
\usepackage{amsmath}
\usepackage{amsfonts}

%opening
\title{Mathematische Knobeleien}
\subtitle{Teil 3 -- Ein Dreieck im Trapez}
\author{Mathematik - Verständlich gemacht!\footnote{Email: \href{mailto:kontakt@mschulte-mathematik.ruhr}{kontakt@mschulte-mathematik.ruhr}}}

\begin{document}

\maketitle

\section*{Problemstellung}
Es sei $T:=ABCD$ ein gleichschenkliges Trapez mit den 
Seiten\footnote{Wir identifizieren Seitennamen mit ihren 
Längen.} $a:=AB$, $b:=BC$, $c:=CD$ und $d:=DA$. Seine Höhe
bezeichnen wir mit $h_T$. Nun ziehen wir die
beiden Diagonalen $AC$ und $DB$ ein und bezeichnen ihren 
Schnittpunkt mit $S$. Von $C$ und $D$ fällen wir jeweils das
Lot auf $a$; die entstehenden Schnittpunkte nennen wir $S_C$ bzw.
$S_D$. Es entsteht ein Dreieck $\Delta := S_DS_CS$. 
Bestimme den Flächeninhalt von $\Delta$.

\section*{Lösung}
xxx

\section*{Ergänzungen}
Nicht-gleichschenklig!

\section*{Quellen}
Die Aufgabe entspringt eigenen Überlegungen.

\end{document}
