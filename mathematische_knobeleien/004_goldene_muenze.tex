\documentclass{scrartcl}
\usepackage{inputenc}
\usepackage[colorlinks=true,urlcolor=blue]{hyperref}
\usepackage{amsmath}
\usepackage{amsfonts}
\usepackage[ngerman]{babel}

%opening
\title{Mathematische Knobeleien}
\subtitle{Teil 4 -- Eine goldene Münze}
\author{Mathematik - Verständlich gemacht!\footnote{Email: \href{mailto:kontakt@mschulte-mathematik.ruhr}{kontakt@mschulte-mathematik.ruhr}}}

\begin{document}

\maketitle

\section*{Problemstellung}
Eine Münze $M$ wird derart gefälscht, dass die 
Wahrscheinlichkeit, mit ihr zweimal hintereinander \textit{Kopf} zu 
werfen, genauso groß ist, wie die Wahrscheinlichkeit, mit $M$ 
\textit{Zahl} zu werfen. Wie groß ist die Wahrscheinlichkeit, mit 
$M$ \textit{Kopf} zu werfen?

\section*{Lösung}
Es sei $\Omega := \{K, Z\}$. Für den Wurf von $M$ sei ein
Wahrscheinlichkeitsmaß $\mathbb{P}$ auf $\Omega$ definiert durch
$$
\mathbb{P}(K) := x, ~ 
\mathbb{P}(Z) := 1-x,
$$
mit $x \in [0,1]$. Wir suchen folglich den Wert von $x$.

\noindent 
Die Wahrscheinlichkeit, zweimal hintereinander \textit{Kopf} zu
werfen ist dann gegeben durch
$$
\mathbb{P}(K) \cdot \mathbb{P}(K) = x^2,
$$
und soll gleich der Wahrscheinlichkeit sein, \textit{Zahl} zu
werfen:
$$
x^2 = 1-x.
$$
Die Lösungen zu dieser Gleichung sind gegeben durch 
$$
x_{1/2} = \frac{-1 \pm \sqrt{5}}{2};
$$
wegen $x \in [0,1]$ folgt also $x = \frac{\sqrt{5}-1}{2}$.

\noindent 
Bezeichnen wir mit $\Phi := \frac{1+\sqrt{5}}{2}$ den 
\textit{goldenen Schnitt}\footnote{
\href{https://de.wikipedia.org/wiki/Goldener_Schnitt}
{Wikipedia-Artikel zum goldenen Schnitt.}}, 
so ist $x = \frac{1}{\Phi}$. 
Daher ist unsere Münze eine \textit{goldene Münze}.

\section*{Quellen}
Die Problemstellung entstammt aus

\centering
\textit{Heinrich Hemme: Heureka! Mathematische Rätsel 2023: Tageskalender mit Lösungen.
	2022. München. Anaconda Verlag. ISBN 978-3-7306-1077-0.}

\raggedright
Es handelt sich um das Rätsel vom 23.03.2023.

\end{document}
